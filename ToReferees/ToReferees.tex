\documentclass[a4paper]{article}
\usepackage[utf8]{inputenc}
\usepackage{fullpage}
\usepackage{threeparttable} % tnote; table footnote
\usepackage{booktabs} % toprule, midrule, bottomrule
\usepackage{multirow} % multirow, multicolumn
\usepackage{float}
\usepackage{xcolor}
\usepackage{amsmath}
\usepackage{enumitem}

%%%%%%%%%%%%%%%%%%%%%%%%%%%%%%%%%%%%%%%%%%%%%%%%%%%%%%%%%%%%%%%%%%%%%%%%%%%%%%%%
% maketitle structure
%%%%%%%%%%%%%%%%%%%%%%%%%%%%%%%%%%%%%%%%%%%%%%%%%%%%%%%%%%%%%%%%%%%%%%%%%%%%%%%%

\makeatletter
\def\id#1{\gdef\@id{#1}}
\def\@id{\@latex@error{No \noexpand\id given}\@ehc}

\renewcommand{\maketitle}{%
\begin{center}\Large\sf
	Authors' response to the review of Manuscript \@id
\end{center}
Manuscript \@id, ``\@title'' \bigskip\par
We appreciate the knowledgeable and helpful comments that we have taken
carefully and did our best to comply with. And this is how we have responded
to each of the referee's comments. \bigskip\par
}
\makeatother
%%%%%%%%%%%%%%%%%%%%%%%%%%%%%%%%%%%%%%%%%%%%%%%%%%%%%%%%%%%%%%%%%%%%%%%%%%%%%%%%


%%%%%%%%%%%%%%%%%%%%%%%%%%%%%%%%%%%%%%%%%%%%%%%%%%%%%%%%%%%%%%%%%%%%%%%%%%%%%%%%
% newcommands
%%%%%%%%%%%%%%%%%%%%%%%%%%%%%%%%%%%%%%%%%%%%%%%%%%%%%%%%%%%%%%%%%%%%%%%%%%%%%%%%

\newcommand*{\Reviewer}[1]{
\noindent {\Large\sf\begin{center}%
	Authors' response to the comments from Referee #1%
\end{center}}}

\newenvironment*{qna}{%
	\newcommand{\comment}{\item \sf}%
	\newcommand{\answer}{\medskip\\Answer: \rm}%
	\newcommand*{\quotehead}{\medskip\par\rm}%
	\newcommand*{\quotetail}{\par\it}%
	\newcommand{\quoteit}{\medskip\par\it}%
	\begin{itemize}%
	}{%
	\end{itemize}\normalfont%
}
\newcommand*{\thead}[1]{\multicolumn{1}{c}{#1}}
\newcommand*{\multithead}[2]{\multicolumn{#1}{@{}c@{}}{\begin{tabular}{c} #2 \end{tabular}}}
%%%%%%%%%%%%%%%%%%%%%%%%%%%%%%%%%%%%%%%%%%%%%%%%%%%%%%%%%%%%%%%%%%%%%%%%%%%%%%%%


%%%%%%%%%%%%%%%%%%%%%%%%%%%%%%%%%%%%%%%%%%%%%%%%%%%%%%%%%%%%%%%%%%%%%%%%%%%%%%%%
% maketitle arguments
%%%%%%%%%%%%%%%%%%%%%%%%%%%%%%%%%%%%%%%%%%%%%%%%%%%%%%%%%%%%%%%%%%%%%%%%%%%%%%%%

% title
\title{Epistasis-based Basis Estimation Method for Simplifying the Problem
	   Space of an Evolutionary Search in Binary Representation}

% id
\id{2095167}
%%%%%%%%%%%%%%%%%%%%%%%%%%%%%%%%%%%%%%%%%%%%%%%%%%%%%%%%%%%%%%%%%%%%%%%%%%%%%%%%


%%%%%%%%%%%%%%%%%%%%%%%%%%%%%%%%%%%%%%%%%%%%%%%%%%%%%%%%%%%%%%%%%%%%%%%%%%%%%%%%
% content of document
%%%%%%%%%%%%%%%%%%%%%%%%%%%%%%%%%%%%%%%%%%%%%%%%%%%%%%%%%%%%%%%%%%%%%%%%%%%%%%%%

\begin{document}
\maketitle

\Reviewer{1}
\begin{qna}
\comment
	The authors never provide a clear definition for what they deem as a
	difficult or easy problem/search space.
\answer
	We agree with the referee that we do not provide a definition for difficult or easy problem/search space. Reflecting the referee's comments, we added a sentence related defintion which difficult of problem/search space to the introduction section as follows:
\quoteit
	Epistasis has the advantage that it is possible to measure the extent of nonlinearity only with fitness function.
	In this paper, we define the difficulty of the problem or problem search space as the nonlinearity level of gene expression. Also, we use epistasis as a measure for the difficulty of the problem.
\end{qna}

\begin{qna}
\comment
	It seems that difficulty is presumed to be captured by epistasis alone, but
	this is not the case. A much longer discussion should be included about
	difficulty in this context, the role that epistatic effects may have in
	determining problem difficulty and the limitations it brings as well,
	including for the particular measure used in this manuscript.
\answer
	Since we did not provide a clear definition of problem difficulty, we think that the referee commented that estimating the difficulty with epistasis alone may be a trouble. So we think that the referee advised us that we need more explanation of the difficulty. For that reason, the answer to the trouble will be resolved by providing a clear definition of the difficulty.

	We agree with the referee that the discussion on the role and limitation of epistasis is needed.
	We added a description of the role of epistasis to the beginning of the difficulty definition.
	We also clearly explained the limit in the end of Section 2.3 as follows:
\quoteit
	Note that nonlinearity may be misleading due to approximation error by solution sampling.
	It hinders to find the proper basis for the target problem.
	The target problem may be transformed into a more complex problem through a basis transformation. That is, the basis transformation can rather prevent a GA from efficiently finding the solution.
\end{qna}

\begin{qna}
\comment
	Often the authors refer to ``better'', ``significantly'', ``good'', ``very
	long'', ``size of the problem'', etc but do not qualify their statement
	with a numerical assessment as well.
\answer
	That is a good point. We confirmed that we did not express the result numerically in Section 6.2, and we added two sentences to the fourth paragraph as follows:
\quoteit
	\begin{enumerate}[label=\alph*)]
	\item Note that when $ n $ is 50, it was over 2 hours.
	\item In particular, when $ n $ is 20, the number of optima found in `Original' is 30, and the numbers of optima found in `Epistasis-sq' and `Epistasis-cu' are 64 and 33, respectively.
	\end{enumerate}
\end{qna}

\begin{qna}
\comment
	S6.2 4th paragraph (``In general, good results...'') is particularly very
	confusing and vague.
\answer
	We agree with the referee. We have made the sentence precise as follows:
\quoteit
	In Table 3, `Meta' found opimal solutions more frequently than the other methods.
\end{qna}

\begin{qna}
\comment
	The tables and figures should have more informative captions that allow the
	reader to better understand what is being presented.
\answer
	We agree with the referee. We tried to put enough explanation in each table and figure caption.	

\quotehead
	Caption of Table~3:
\quotetail
	Results of each of the best solutions obtained by conducting the GA experiments 100 times on an instance of the variant-onemax problem. (`\# of optima' is the number of optima found during 100 experiments, `Average' is the average of 100 best solutions, and `SD' is the standard deviation of 100 best solutions. $ Q_1, \, Q_2, $ and $ Q_3 $ are the first, second, and third quartiles, respectively. `Time' is the sum of the time to search for the basis and that for the GA experiments.)

\quotehead
	Caption of Figure~3: \par
\quotetail
	A box plot of each of the best solutions obtained by conducting the GA experiment 100 times on an instance of the variant-onemax problem.

\quotehead
	Caption of Table~4: \par
\quotetail
	Results of each of the best solutions obtained by conducting the GA experiments 100 times on an instance of the $ NK $-landscape problem. (`Best' is the best fitness among solutions found in 100 experiments, `Average' is the average of 100 best solutions, and `SD' is the standard deviation of 100 best solutions. $ Q_1, \, Q_2, $ and $ Q_3 $ are the first, second and third quartiles, respectively. `Time' is the sum of the time to search for the basis and that for the GA experiments.)

\quotehead
	Caption of Figure~4:
\quotetail
	A box plot of each of the best solutions obtained by conducting the GA experiment 100 times on an instance of the $ NK $-landscape problem.
\end{qna}

\begin{qna}
\comment
	Many of the results focus on best found solution from the 100 trials.
	However, why is the median or distribution of final results not included?
	It would seem quite important to provide such data since the expected
	performance is more what should be ... expected... and gets more to the
	question of ``are we likely to get a better outcome if we use a change of
	basis approach''?
\answer
	That is a good point. We think that there was a lack of distribution data for expected performance. We have changed Tables~3 and 4 with more statistical information as follows:

\begin{table}[H]
	\centering\footnotesize
	\renewcommand\thetable{3} 
	\caption{ Results of each of the best solutions obtained by conducting the GA experiments 100 times on an instance of the variant-onemax problem. (`\# of optima' is the number of optima found during 100 experiments, `Average' is the average of 100 best solutions, and `SD' is the standard deviation of 100 best solutions. $ Q_1, \, Q_2, $ and $ Q_3 $ are the first, second, and third quartiles, respectively. `Time' is the sum of the time to search for the basis and that for the GA experiments.)} \label{tab:result_var}
	\vspace*{0.2cm}
	\begin{threeparttable}
		\begin{tabular}{ccccccccc}
			\toprule
			$ n $ & Type & \# of optima & Average & SD & $ Q_1 $ & $ Q_2 $ & $ Q_3 $ &  \begin{tabular}{@{}c@{}} Time \\ (mm:ss)\tnote{*} \end{tabular} \\
			\midrule
			\multirow{4}{*}{20} & Original		& 30 & 0.945 & 0.0452 & 0.900 & 0.950 & 1.000 & { } { } 0:44 \\
			& Meta			& 66 & 0.980 & 0.0302 & 0.950 & 1.000 & 1.000 & { } { } 3:07 \\
			& Epistasis-sq	& 64 & 0.982 & 0.0241 & 0.950 & 1.000 & 1.000 & { } { } 1:01 \\
			& Epistasis-cu	& 33 & 0.964 & 0.2760 & 0.950 & 0.950 & 1.000 & { } { } 3:11 \\
			\midrule
			\multirow{4}{*}{30} & Original		& 31 & 0.963 & 0.0329 & 0.930 & 0.970 & 1.000 & { } { } 1:09  \\
			& Meta			& 82 & 0.993 & 0.0155 & 1.000 & 1.000 & 1.000 & { } 12:15 \\
			& Epistasis-sq	& 47 & 0.979 & 0.0216 & 0.967 & 0.967 & 1.000 & { } { } 3:49  \\
			& Epistasis-cu	& 40 & 0.979 & 0.0187 & 0.967 & 0.967 & 1.000 & { } { } 7:03  \\
			\midrule
			\multirow{4}{*}{50} & Original		& 0  & 0.931 & 0.0257 & 0.920 & 0.940 & 0.940 & { } { } 2:58   \\
			& Meta			& 0  & 0.939 & 0.0240 & 0.920 & 0.940 & 0.960 & 136:46 \\
			& Epistasis-sq	& 2	 & 0.934 & 0.0272 & 0.920 & 0.940 & 0.945 & { } { } 7:48   \\
			& Epistasis-cu	& 0	 & 0.927 & 0.0272 & 0.900 & 0.920 & 0.940 & { } 67:59  \\	
			\bottomrule
		\end{tabular}
		\begin{tablenotes}
			\footnotesize
			\item[*] On Intel (R) Core TM i7-6850K CPU @ 3.60GHz
		\end{tablenotes}
	\end{threeparttable}
\end{table}

\begin{table}[H]
	\centering
	\renewcommand\thetable{4} \footnotesize
	\caption{Results of each of the best solutions obtained by conducting the GA experiments 100 times on an instance of the $ NK $-landscape problem. (`Best' is the best fitness among solutions found in 100 experiments, `Average' is the average of 100 best solutions, and `SD' is the standard deviation of 100 best solutions. $ Q_1, \, Q_2, $ and $ Q_3 $ are the first, second, and third quartiles, respectively. `Time' is the sum of the time to search for the basis and that for the GA experiments.)} \label{tab:result_nk}
	\vspace*{0.2cm}
	\begin{threeparttable}
	\begin{tabular}{ccccccccr@{:}l}
	\toprule
	$ N,\, K$ & Type & Best & Average & SD & $ Q_1 $ & $ Q_2 $ & $ Q_3 $ & \multithead{2}{Time \\ (mm:ss)\tnote{*}} \\
	\midrule
	\multirow{3}{*}{20, 3}	& Original  & 0.817 & 0.8135 & 0.0085 & 0.8170 & 0.8170 & 0.8170 & 1 & 02 \\
	& Meta	    & 0.825 & 0.8226 & 0.0057 & 0.8250 & 0.8250 & 0.8250 & 5 & 52 \\
	& Epistasis & 0.825 & 0.8200 & 0.0056 & 0.8170 & 0.8170 & 0.8250 & 1 & 32 \\
	\midrule
	\multirow{3}{*}{20, 5}	& Original  & 0.761 & 0.7449 & 0.0157 & 0.7400 & 0.7405 & 0.7610 & 1 & 03 \\
	& Meta	    & 0.761 & 0.7533 & 0.0131 & 0.7470 & 0.7610 & 0.7610 & 5 & 39 \\
	& Epistasis & 0.761 & 0.7505 & 0.0109 & 0.7460 & 0.7470 & 0.7610 & 1 & 40 \\
	\midrule
	\multirow{3}{*}{20, 10}	& Original  & 0.779 & 0.7306 & 0.0253 & 0.7020 & 0.7335 & 0.7520 & 1 & 10 \\
	& Meta      & 0.785 & 0.7572 & 0.0155 & 0.7660 & 0.7550 & 0.7660 & 7 & 13 \\
	& Epistasis & 0.785 & 0.7558 & 0.0136 & 0.7460 & 0.7530 & 0.7653 & 2 & 16 \\
	\midrule
	\multirow{3}{*}{30, 3}	& Original  & 0.776 & 0.7687 & 0.1373 & 0.7740 & 0.7760 & 0.7760 & 2 & 06 \\
	& Meta      & 0.776 & 0.7719 & 0.0109 & 0.7760 & 0.7760 & 0.7760 & 5 & 39 \\
	& Epistasis & 0.776 & 0.7718 & 0.0090 & 0.7740 & 0.7760 & 0.7760 & 1 & 40 \\
	\midrule
	\multirow{3}{*}{30, 5}	& Original  & 0.795 & 0.7725 & 0.0125 & 0.7638 & 0.7740 & 0.7870 & 2 & 06 \\
	& Meta      & 0.795 & 0.7661 & 0.0170 & 0.7540 & 0.7710 & 0.7770 & 32 & 28 \\
	& Epistasis & 0.795 & 0.7706 & 0.0136 & 0.7623 & 0.7730 & 0.7830 & 2 & 50 \\
	\midrule
	\multirow{3}{*}{30, 10}	& Original  & 0.779 & 0.7349 & 0.0181 & 0.7260 & 0.7310 & 0.7443 & 2 & 06 \\
	& Meta      & 0.805 & 0.7391 & 0.0179 & 0.7310 & 0.7370 & 0.7470 & 49 & 47 \\
	& Epistasis & 0.796 & 0.7366 & 0.0198 & 0.7220 & 0.7335 & 0.7960 & 3 & 48 \\
	\midrule
	\multirow{3}{*}{30, 20}	& Original  & 0.750 & 0.7039 & 0.0152 & 0.6938 & 0.7010 & 0.7113 & 2 & 51 \\
	& Meta      & 0.762 & 0.7181 & 0.0163 & 0.7070 & 0.7155 & 0.7243 & 49 & 47 \\
	& Epistasis & 0.770 & 0.7220 & 0.0133 & 0.7120 & 0.7200 & 0.7300 & 3 & 48 \\
	\midrule
	\multirow{3}{*}{50, 3}	& Original  & 0.776 & 0.7576 & 0.0102 & 0.7515 & 0.7590 & 0.7640 & 5 & 31 \\
	& Meta      & 0.776 & 0.7599 & 0.0119 & 0.7530 & 0.7585 & 0.7730 & 220 & 14 \\
	& Epistasis & 0.776 & 0.7578 & 0.0096 & 0.7508 & 0.7590 & 0.7630 & 6 & 34 \\
	\bottomrule
\end{tabular}
		\begin{tablenotes}
			\footnotesize
			\item[*] On Intel (R) Core TM i7-6850K CPU @ 3.60GHz
		\end{tablenotes}
	\end{threeparttable}
\end{table}
\end{qna}

\begin{qna}
\comment
	Table 5 and 6 show data for increasing $ n $ and before/after. This gap seems to be closing as n increases, and the authors should comment on this, and perhaps explain/prove some convergence (if one exists).
\answer
	It does not seem to decrease the \textit{gap} as $ n $ increases. 
	For the helping to understand the \textit{gap}, we added decrease rates to Tables~5 and 6 as follows:

\begin{table}[H]
	\renewcommand\thetable{5} 
	\caption{Epistasis of the original and modified basis sampling in the variant-onemax problem.} \label{tab:epi_var}
	\vspace*{0.2cm}
	\centering
	\begin{threeparttable}
		\begin{tabular}{ccccc}
			\toprule
			\multirow{2.5}{*}{$ n $} & \multirow{2.5}{*}{Sampling size} & \multicolumn{3}{c}{Epistasis} \\	\cmidrule(lr){3-5}
			
			& & Before & After & Decrease rate (\%)\tnote{*}       \\
			\midrule
			\multirow{2}{*}{20} & square	& 4.46 & 3.23 & 27.6   \\
			& cubic		& 4.35 & 3.83 & 12.0   \\
			\midrule
			\multirow{2}{*}{30} & square	& 4.57 & 3.20 & 30.0   \\
			& cubic		& 5.00 & 3.72 & 25.6   \\
			\midrule
			\multirow{2}{*}{50} & square	& 9.27 & 7.53 & 18.8   \\
			& cubic		& 9.69 & 8.93 & { 7.8} \\
			\bottomrule
		\end{tabular}
		\begin{tablenotes}
			\footnotesize
			\item[*] $ \text{Decrease rate} = 100 \times \left( \text{Before} - \text{After} \right) / \text{Before}$
		\end{tablenotes}
	\end{threeparttable}
\end{table}

\begin{table}[H]
	\renewcommand\thetable{6} 
	\caption{Epistasis of the original and modified basis sampling in the $NK$-landscape problem.} \label{tab:epi_nk}
	\vspace*{0.2cm}
	\centering
	\begin{threeparttable}
		\begin{tabular}{r@{, }lccc}
			\toprule
			\multicolumn{2}{c}{\multirow{2.5}{*}{$ N,\, K $}} & \multicolumn{3}{c}{Epistasis} \\	\cmidrule(lr){3-5}
			\thead{} & & Before & After & Decrease rate (\%)\tnote{*} \\
			\midrule
			20 & 3	& $ 3.17e^{-3} $ & $ 2.25e^{-3} $ & 29.0    \\
			20 & 5	& $ 3.16e^{-3} $ & $ 2.90e^{-3} $ & { 8.2}  \\
			20 & 10	& $ 4.28e^{-3} $ & $ 3.82e^{-3} $ & 10.7    \\
			30 & 3	& $ 1.85e^{-3} $ & $ 1.60e^{-3} $ & 13.5    \\
			30 & 5	& $ 2.61e^{-3} $ & $ 2.37e^{-3} $ & { 9.2}  \\
			30 & 10	& $ 2.68e^{-3} $ & $ 2.39e^{-3} $ & 10.8    \\
			30 & 20	& $ 2.78e^{-3} $ & $ 2.50e^{-3} $ & 10.1    \\
			50 & 3	& $ 1.13e^{-3} $ & $ 9.32e^{-4} $ & 17.5    \\
			\bottomrule
		\end{tabular}
		\begin{tablenotes}
			\footnotesize
			\item[*] $ \text{Decrease rate} = 100 \times \left( \text{Before} - \text{After} \right) / \text{Before}$
		\end{tablenotes}
	\end{threeparttable}
\end{table}
\end{qna}


\newpage
\Reviewer{3}
\begin{qna}
\comment
	This paper proposes a technique for changing the basis of the underlying
	problem representation, thereby reducing the epistasis. In essence, this
	allows a more challenging problem to be transformed into an easier problem,
	thereby allowing a more effective search, by way of genetic algorithm, to
	be carried out. Overall, the paper is well written and easy to follow. In
	my opinion, this is an excellently written paper that is clear in its
	methodology and the results are sound. 
\answer
Thanks a lot for the good summaray of the paper. We did our best to improve the paper based on the reviewer's constructive comments.

\end{qna}

\begin{qna}
\comment
	Perhaps it would be beneficial to state that the proposed process aims to
	transform a non-separable problem to a separable problem. This terminology
	may target a wider audience and gives a much clearer indication of the
	overall effect that chaning the basis can have.
\answer
	Thanks for the constructive comments. Reflecting the referee's comments, we added a sentence of the statement to the introduction section as follows:
\quoteit
	Our intention in this study is that a non-separable problem can be transformed into a separable problem by performing an appropriate basis transformation. Such an altered environment enables GA to search space effectively.
\end{qna}

\begin{qna}
\comment
	Page 3 - it is good to include a header paragraph before the first
	subsection. Thus, you should include a paragraph at the beginning of
	Section 3 that describes what the overall purpose of Section 3 is.
\answer
	Thanks for the good comment. Reflecting the referee's comments, we added a paragraph to the beginning of Section 3 as follows:
\quoteit
	This section presents a GA that performs an effective search through a change of basis. Before presenting the GA, we introduce the related terminologies and theories of change of basis in binary representation.
	Next, we apply the change of basis in the onemax problem to show how the problem actually transformed. In addition, a methodology for evaluating solutions in the transformed problem will be described.
	Finally, we propose a GA that effectively searches solutions through applying the change of basis.
	On the other hand, searching for an appropriate basis will be covered in Sections~4 and 5.
\end{qna}

\begin{qna}
\comment 
	Algorithm 1 - if $ P^\prime $ and $ O^\prime $ are used to generate the new
	population, should there not be a process in this algorithm that translates
	$ P^\prime \to P $? Otherwise, how is fitness calculated using the new
	representation? Perhaps Step 6 should read `the process from Step 2 onward'
	rather than `Step 3 onward'? Also, would it not make sense to return the
	population in the original basis, rather than the new basis? Otherwise, the
	GA is returning a solution with a different encoding than it was initially
	set to optimize. I find that Algorithm 1 should be edited for clarity as it
	provides the foundation for the remainder of the work and any uncertainty
	about the process stemming from here would likely detract from the overall
	impact of the proposed technique.
\answer
	As the referee said, the process from $ P^\prime \to P $ must be at the end. Reflecting the referee's comments, we changed Step 6 of Algorithm 1 as follows:
\quoteit
\begin{itemize}
	\item[Step 6:] the process from Step 3 onward is repeated as many times as there are generations. When the number of generations has been exceeded, then
	we return $ P^\prime $ whereby the basis $ B $ is changed to the standard basis $ B_s $.
\end{itemize}
\end{qna}


Thank you again.
\end{document}
%%%%%%%%%%%%%%%%%%%%%%%%%%%%%%%%%%%%%%%%%%%%%%%%%%%%%%%%%%%%%%%%%%%%%%%%%%%%%%%%